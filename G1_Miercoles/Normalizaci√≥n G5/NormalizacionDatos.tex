\documentclass[12pt]{article}
\usepackage{graphicx} % Para incluir imágenes
\usepackage{amsmath} % Para expresiones matemáticas
\usepackage{hyperref} % Para enlaces

% Datos del documento
\title{Normalización de Datos}

\author{[Diego Álzate - Aris Ávila - Julieth Gutierrez]}

\date{\today}

\begin{document}

\maketitle

\begin{abstract}
    En este documento se presenta la importancia de la normalización de datos en el análisis de información, detallando el proceso aplicado y sus beneficios.
\end{abstract}

\section{Introducción}
La normalización de datos es un paso crucial en el procesamiento y análisis de información, especialmente cuando se trabaja con grandes volúmenes de datos que presentan diferentes escalas o formatos. En este trabajo se aborda el proceso de normalización aplicado a un conjunto de datos específico, detallando sus beneficios y mostrando ejemplos prácticos de su implementación.

\section{Desarrollo del Proceso}
Para llevar a cabo la normalización de los datos, se realizaron los siguientes pasos:
\begin{enumerate}
    \item Identificación de los datos sin normalizar.
    \item Aplicación de técnicas de normalización.
    \item Comparación entre los datos originales y los normalizados.
\end{enumerate}

\subsection{Datos sin Normalizar}
A continuación, se muestra un ejemplo de cómo estaban los datos antes del proceso de normalización:

\begin{figure}[h]
    \centering
    \includegraphics[width=0.8\textwidth]{datos_sin_normalizar.png}
    \caption{Ejemplo de datos sin normalizar.}
    \label{fig:sin_normalizar}
\end{figure}

\subsection{Datos Normalizados}
Tras aplicar el proceso de normalización, los datos quedaron representados de la siguiente manera:

\begin{figure}[h]
    \centering
    \includegraphics[width=0.8\textwidth]{datos_normalizados.png}
    \caption{Ejemplo de datos normalizados.}
    \label{fig:normalizados}
\end{figure}

\section{Importancia de la Normalización de Datos}
La normalización es fundamental para garantizar la calidad y coherencia de los datos. Algunos de sus principales beneficios incluyen:
\begin{itemize}
    \item Mejora la precisión de los modelos de análisis de datos.
    \item Reduce redundancias y errores.
    \item Facilita la integración de datos provenientes de diferentes fuentes.
\end{itemize}

\section{Conclusión}
La normalización de datos es un paso esencial en el procesamiento de información, permitiendo mejorar la calidad y la precisión de los análisis. En este trabajo se evidenció cómo un conjunto de datos puede transformarse mediante la normalización y los beneficios que esto aporta en la toma de decisiones.

\end{document}
