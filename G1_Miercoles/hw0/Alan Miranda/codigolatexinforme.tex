\documentclass[a4paper, 10pt]{article}
\hyphenpenalty=8000
\textwidth=125mm
\textheight=185mm

\usepackage{graphicx}
% Para la inclusión de imágenes
\usepackage{alltt}
% Para la descripción de algoritmos y programas
\usepackage{amsmath}
% Para símbolos matemáticos
\usepackage[hidelinks, pdftex]{hyperref}
% Para enlaces hipermedia
\usepackage{listings}
\usepackage{xcolor}
% Para código y resaltado

\lstset{
  language=Python,
  basicstyle=\small\ttfamily,
  keywordstyle=\color{blue},
  commentstyle=\color{gray},
  stringstyle=\color{red},
  numbers=left,
  numberstyle=\tiny,
  stepnumber=1,
  numbersep=10pt,
  breaklines=true,
  breakatwhitespace=false,
  tabsize=4,
  showstringspaces=false
}

\pagenumbering{arabic}
\setcounter{page}{1}
\renewcommand{\thefootnote}{\fnsymbol{footnote}}

\begin{document}

\begin{center}
\textbf{Estimación del Área de una Mancha Taller Individual}\\[6pt]
\small
\textbf{Alan Miranda}\\[6pt]
Infotep \\[6pt]
Fecha: 28 de febrero de 2025
\end{center}

\begin{abstract}
En este trabajo se estima el área de una mancha presente en una imagen, utilizando un método basado en la generación aleatoria de puntos (método de Monte Carlo) implementado en Python. Se generan puntos aleatorios sobre la imagen y, a partir de la proporción de aquellos que caen en la mancha (definida mediante un umbral de intensidad), se calcula una aproximación del área. 
\end{abstract}

\section{Procedimiento}
El método seguido para la estimación del área se resume en los siguientes pasos:
\begin{enumerate}
    \item \textbf{Carga de la imagen:} Se utiliza la librería \texttt{PIL} para cargar la imagen y convertirla a escala de grises, obteniéndose su ancho (\(w\)) y alto (\(h\)).
    \item \textbf{Generación de puntos:} Se generan \(n\) puntos aleatorios dentro de los límites de la imagen.
    \item \textbf{Determinación de puntos en la mancha:} Para cada punto se evalúa la intensidad de su píxel. Si la intensidad es menor a 128, se considera que el punto cae dentro de la mancha.
    \item \textbf{Cálculo del área:} Se estima el área de la mancha usando la fórmula:
    \[
    \text{Área} = \left(\frac{p}{n}\right) \times (w \times h)
    \]
    donde \(p\) es el número de puntos que caen dentro de la mancha.
\end{enumerate}



\section{Resultados}
Se realizaron pruebas con tres diferentes valores de \( n \):
\begin{itemize}
    \item \textbf{\( n = 1000 \):} Al utilizar un gran número de puntos, la estimación del área se vuelve más estable y precisa.
    \item \textbf{\( n = 100 \):} Se observa una variabilidad moderada en los resultados, aunque los valores son consistentes.
    \item \textbf{\( n = 10 \):} Con pocos puntos, la estimación presenta mayor variabilidad y menor precisión.
\end{itemize}
En general, se concluye que al aumentar el número de puntos aleatorios, la aproximación del área se vuelve más confiable.

\section{Visualización de la Mancha}
La imagen empleada para el estudio es la siguiente:

\includegraphics[width=0.4\linewidth]{mancha_png_by_ona_smile-d5tmk7x.png}

\section{Conclusión}
El método de Monte Carlo implementado en Python permite estimar el área de una mancha en una imagen de forma sencilla y eficaz. Los resultados indican que, aumentando el número de puntos aleatorios, se obtiene una aproximación más precisa del área real. Se recomienda utilizar un número elevado de puntos para minimizar la variabilidad de la estimación.

\end{document}
