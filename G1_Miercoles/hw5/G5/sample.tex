\documentclass[a4paper, 10pt]{article}
\hyphenpenalty=8000
\textwidth=125mm
\textheight=185mm

\usepackage{graphicx}
%this package is flexible for image insertion
\usepackage{alltt}
%this package is suitable for the description of algorithms and computer programs
\usepackage{amsmath}
%this package draws mathematical symbols smoothly
\usepackage[hidelinks, pdftex]{hyperref}
%this package produces hypertext links in the document
\usepackage{float} % Para control de posición de imágenes

\pagenumbering{arabic}
\setcounter{page}{1}
\renewcommand{\thefootnote}{\fnsymbol{footnote}}
\newcommand{\doi}[1]{\href{https://doi.org/#1}{\texttt{https://doi.org/#1}}}

\begin{document}

\begin{center}
Nonlinear Analysis: Modelling and Control, Vol. vv, No. 03, 2025\\
\copyright\IE Infotep\\[24pt]
\LARGE
\textbf{Cuantización de imagenes y estrategia para determinar K y los centroide}\\[6pt]
\small
\textbf {Diego Fernando Alzate Alvarez}\\[6pt]
IE Infotep \\ diegoalzate2014@gmail.com\\[6pt]
\end{center}

\begin{abstract}
Este informe presenta el análisis y resultados obtenidos al aplicar la de cuantización de imágenes de Ciénaga Magdalena. donde se probaron varios numeros de clusters, hasta obtener un buen resultado. Además se propone una estrategia personal de como inicializar los valor de k y los centroides. \vskip 2mm
\end{abstract}

\section{Introducción}
Se realizo cuantización de imágenes, que es un proceso de reducir el número de colores en una imagen mientras se mantiene su estructura visual. El proceso se realizo utilizando el codigo compartido en clase por el profesor, el cual usa el algoritmo de k-means, donde se recibe los clusters (cantidad de colores) a tener la imagen. Por otra parte en este informe, veremos los resultado obtenidos al aplicar la cuantización de imágenes y se propone una estrategia para determinar el valor óptimo de k y la asignación de centroides.

\section{Resultados obtenidos de la cuantización de las imágenes}
Se aplicó el algoritmo de k-means para realizar la cuantización de imágenes con diferentes valores de k, que en este caso es igual a n clusters. A continuación, se presentan los resultados de la cuantización en las siguientes imágenes:

\begin{figure}[H]
    \centering
    \begin{minipage}[b]{0.45\linewidth}
        \centering
        \includegraphics[width=\linewidth]{p3.jpeg}
        \caption{Resultado de la cuantización con k=3}
    \end{minipage}
    \hspace{0.5cm}
    \begin{minipage}[b]{0.45\linewidth}
        \centering
        \includegraphics[width=\linewidth]{p4.jpeg}
        \caption{Resultado de la cuantización con k=4}
    \end{minipage}
    \vskip\baselineskip
    \begin{minipage}[b]{0.45\linewidth}
        \centering
        \includegraphics[width=\linewidth]{p5.jpeg}
        \caption{Resultado de la cuantización con k=5}
    \end{minipage}
    \hspace{0.5cm}
    \begin{minipage}[b]{0.45\linewidth}
        \centering
        \includegraphics[width=\linewidth]{p6.jpeg}
        \caption{Resultado de la cuantización con k=6}
    \end{minipage}
    \vspace{5mm} % Espacio entre las imágenes y la observación
    \textit{Observación: luego de analisar los resulado con varias imágenes y, he llegado a la conclusión de que se necesita un mínimo de 4 clusters para obtener un resultado decente. a lo que me refiero, poder identificar la mayoría de los objetos, personas y demás elementos que puedan estar presentes en las imágenes.}
\end{figure}

\begin{figure}[H]
    \centering
    \begin{minipage}[b]{0.45\linewidth}
        \centering
        \includegraphics[width=\linewidth]{pl3.jpeg}
        \caption{Resultado de la cuantización con k=3}
    \end{minipage}
    \hspace{0.5cm}
    \begin{minipage}[b]{0.45\linewidth}
        \centering
        \includegraphics[width=\linewidth]{pl4.jpeg}
        \caption{Resultado de la cuantización con k=4}
    \end{minipage}
    \vskip\baselineskip
    \begin{minipage}[b]{0.45\linewidth}
        \centering
        \includegraphics[width=\linewidth]{pl5.jpeg}
        \caption{Resultado de la cuantización con k=5}
    \end{minipage}
    \hspace{0.5cm}
    \begin{minipage}[b]{0.45\linewidth}
        \centering
        \includegraphics[width=\linewidth]{pl6.jpeg}
        \caption{Resultado de la cuantización con k=6}
    \end{minipage}
    \vspace{5mm} % Espacio entre las imágenes y la observación
    \textit{Observación: Al igual que en las imágenes anteriores, el número de clusters influye en la claridad de los detalles en la imagen. A k=6, la imagen tiene más colores y es más detallada.}
\end{figure}

\section{Estrategia para determinar el valor de k y la asignación de los centroides}
Para determinar el valor óptimo de k, en una clase se menciono que estos van a depender o estar arraigado a cuantos grupos estemos buscando. En clase se ponia en contexto agrupar un grupo de persona en tres categoria: 

\begin{itemize}
    \item \textbf{Compulsiva para comprar} (Grupo 1)
    \item \textbf{Saber que publicidad ponerle} (Grupo 2)
    \item \textbf{No compran} (Grupo 3)
\end{itemize}

Por otro lado, para la selección de los centroides, propongo asignarlos de manera uniforme al principio. Por ejemplo, si se tienen dos centroides, podrían ocupar el 50\% del espacio cada uno. Si se tienen tres o más centroides, el espacio total se distribuiría proporcionalmente entre ellos.

\end{document}
