\documentclass[a4paper, 10pt]{article}
\usepackage{graphicx}
\usepackage{alltt}
\usepackage{amsmath}
\usepackage[hidelinks, pdftex]{hyperref}

\begin{document}

\title{Implementaci\'on de Algoritmo K-Means sin Librer\'ias}
\author{ROBBYEL ELIAS, CARLOS JERONIMO, ANDRUS LOPEZ}
\date{\today}
\maketitle

\begin{abstract}
Este documento describe la implementaci\'on del algoritmo K-Means sin el uso de librer\'ias en Python. Se explican los pasos principales del algoritmo y se proporciona una implementaci\'on sencilla y comentada del mismo.
\end{abstract}

\section{Introducci\'on}
El algoritmo K-Means es un m\'etodo de agrupamiento utilizado para dividir un conjunto de datos en $k$ grupos o clusters. Su objetivo es minimizar la distancia intra-cluster y maximizar la distancia inter-cluster. En esta implementaci\'on, se presenta una versi\'on simple sin el uso de librer\'ias externas.

\section{Descripci\'on del Algoritmo}
El algoritmo sigue los siguientes pasos:
\begin{enumerate}
    \item Seleccionar $k$ puntos iniciales como centroides.
    \item Asignar cada punto de datos al centroide m\'as cercano.
    \item Recalcular los centroides como el promedio de los puntos asignados.
    \item Repetir hasta que los centroides no cambien significativamente.
\end{enumerate}

\section{Implementaci\'on en Python}
A continuaci\'on se presenta el c\'odigo en Python con explicaciones detalladas:

\begin{alltt}
# Definimos los datos de entrada
# Cada punto es representado por sus coordenadas (x, y)
datos = [
    [1, 2], [2, 1], [1.5, 1.8],
    [5, 8], [6, 9], [6.5, 8.5],
    [1, 0.6], [9, 11], [8, 10],
    [3, 4], [2.5, 3.5], [3.2, 3.8]
]

# Número de grupos
k = 3

# Función para calcular la distancia euclidiana entre dos puntos
def distancia(a, b):
    return ((a[0] - b[0]) * 2 + (a[1] - b[1]) * 2) ** 0.5

# Inicializamos los centroides con los primeros k puntos
centroides = datos[:k]

# Número de iteraciones fijas
for iteracion in range(10):
    asignaciones = []  # Lista para guardar el cluster de cada punto
    
    # Asignamos cada punto al centroide más cercano
    for punto in datos:
        distancias = [distancia(punto, centro) for centro in centroides]
        cluster = distancias.index(min(distancias))
        asignaciones.append(cluster)
    
    # Recalcular los centroides
    nuevos_centroides = [[0, 0] for _ in range(k)]
    conteo = [0] * k
    
    for indice, punto in enumerate(datos):
        grupo = asignaciones[indice]
        nuevos_centroides[grupo][0] += punto[0]
        nuevos_centroides[grupo][1] += punto[1]
        conteo[grupo] += 1
    
    for i in range(k):
        if conteo[i] != 0:
            nuevos_centroides[i][0] /= conteo[i]
            nuevos_centroides[i][1] /= conteo[i]
        else:
            nuevos_centroides[i] = centroides[i]
    
    centroides = nuevos_centroides

    print(f"Iteración {iteracion + 1} - Centroides: {centroides}")

print("Asignación final de puntos a clusters:", asignaciones)
\end{alltt}

\section{Conclusiones}
Esta implementaci\'on del algoritmo K-Means permite agrupar datos sin necesidad de librer\'ias especializadas. Se logra con un c\'odigo sencillo y humanizado que puede adaptarse a diferentes aplicaciones.

\end{document}